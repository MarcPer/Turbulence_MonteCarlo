\documentclass[12pt]{article}
\usepackage[utf8]{inputenc}
\usepackage{amsmath,amsfonts,amssymb,multirow,indentfirst,graphicx}

\newcommand{\inner}[2]{\ensuremath{\left( #1 \, , \, #2 \right)}}
\newcommand{\classoption}[1]{\texttt{#1}}
\newcommand{\macro}[1]{\texttt{\textbackslash#1}}
\newcommand{\m}[1]{\macro{#1}}
\newcommand{\env}[1]{\texttt{#1}}
\newcommand{\abs}[1]{\ensuremath{\left| #1 \right|}}
\newcommand{\media}[1]{\ensuremath{\big\langle #1 \big\rangle}}
\newcommand{\vect}[1]{\ensuremath{\boldsymbol{\mathrm{#1}}}}
\newcommand{\sinc}[1]{\ensuremath{\mathrm{sinc}\, #1}}
\newcommand{\ket}[1]{\ensuremath{|#1\rangle}}
\newcommand{\bra}[1]{\ensuremath{\langle #1|}}
\newcommand{\braket}[2]{\langle #1|#2\rangle}
\newcommand{\ave}[1]{\langle #1\rangle}


\title{Turbulence Monte Carlo}
\author{M. V. da Cunha Pereira}
\date{\today}

\begin{document}
  \maketitle 
\begin{abstract}
This document describes some of the theory behind simulations performed in this project and discusses the sampling constraints that ought to be verified in order for the results to be valid.
\end{abstract}

\section{From Fourth to Second-Order Simulation}
Performing simulations for quantities that are fourth-order in field is a formidable task when propagation under atmospheric turbulence is considered, so we try to reframe the model in terms of equivalent second-order beam propagation. This can be done for special cases detailed below, using, for the phase screens that emulate turbulent refractive index fields, a modified 'effective field'.

\subsection{Delta Correlation in Near Field}
We first consider a special case in which the fourth-order beam is delta correlated in the near-field, so that on the source-plane it has the form
\begin{align}
E_0(\rho_1^\prime, \rho_2^\prime) = E_p\left(\frac{\rho_1^\prime\pm \rho_2^\prime}{2}\right) &V_a(\rho_1^\prime\mp \rho_2^\prime) \hspace{1cm} \\
&\mbox{, where } V_a(\rho) = \delta^2(\rho-a) \, . \notag
\end{align}

Top and bottom sign refer to the case where there is and isn't a coordinate inversion scheme in place. We consider the former for concreteness and as it corresponds to the more interesting case. Hence its propagation reads:
\begin{align}
E_a(\rho_1,\rho_2) &= \iint E_p\left(\frac{\rho_1^\prime-\rho_2^\prime}{2}\right) V_a(\rho_1^\prime+\rho_2^\prime) h(\rho_1,\rho_1^\prime) h(\rho_2, \rho_2^\prime) \; d^2\rho_1^\prime d^2\rho_2^\prime \notag \\
&= \int E_p\left(\rho_1^\prime+\frac{a}{2}\right) h(\rho_1,\rho_1^\prime) h(\rho_2, -\rho_1^\prime+a) \; d^2\rho_1^\prime \,  \notag \\
&= \int E_p(\rho^\prime)h\left(\rho_1, \rho^\prime+\frac{a}{2}\right) h\left(\rho_2, -\rho^\prime+\frac{a}{2}\right) \;  d^2\rho^\prime.
\end{align}
The propagators $h(x,y)$ can be split in the following form:
\begin{align}
h(x,y) = &h_0(x,y)\exp{\left[ \psi(x,y) \right]} \\
& \mbox{, where } h_0(x,y) \propto \exp{\left[\frac{ik(x-y)^2}{2z}\right]} \label{eq_vaccuum-propagator}\; .
\end{align}
The term $h_0$ is the usual vaccuum propagator, while $\psi$, which describes the contribution from refractive index turbulence, is written
\begin{align}
\psi(\rho,\rho^\prime) \propto \int_0^L dz^\prime &\int d^2\rho'' \frac{n_1(\rho'',z')}{z'(L-z')} \notag \\
& \times \exp{ \left\{ \frac{ik}{2} \left[ \frac{(\rho-\rho'')^2}{L-z'} + \frac{(\rho''-\rho')^2}{z'} - \frac{(\rho-\rho')^2}{L}\right] \right\}} \; .
\end{align}
Close inspection of \eqref{eq_vaccuum-propagator} reveals that
\begin{align}
h_0(\rho_1,\rho'+a/2) & = h_0(\rho_1-a/2,\rho') \\
h_0(\rho_2,-\rho'+a/2)& = h_0(-\rho_2+a/2,\rho') \; .
\end{align}
As for the contribution from atmospheric turbulence, one can see that:
\begin{align}
\psi\left(\rho_1,\rho^\prime+\frac{a}{2}\right) &\propto \int_0^L dz^\prime \int d^2 \rho^{\prime\prime} \frac{n_1(\rho^{\prime\prime},z^\prime)}{z^\prime (L-z^\prime)} \ldots \notag \\
& \hspace{1cm} \times \exp{\left\{\frac{ik}{2}\left[ \frac{(\rho_1-\rho^{\prime\prime})^2}{L-z^\prime} + \frac{\left(\rho^{\prime\prime}-\rho^\prime-\frac{a}{2}\right)^2}{z^\prime} - \frac{\left(\rho_1 -\rho^\prime -\frac{a}{2}\right)^2}{L}\right]\right\}} \notag \\
&=\int_0^L dz^\prime \int d^2 \tau \frac{n_1\left(\tau+\frac{a}{2},z^\prime\right)}{z^\prime (L-z^\prime)} \ldots \notag \\
& \hspace{1cm} \times \exp{\left\{\frac{ik}{2}\left[ \frac{(\rho_1- \frac{a}{2}-\tau)^2}{L-z^\prime} + \frac{\left(\tau-\rho^\prime\right)^2}{z^\prime} - \frac{\left(\rho_1 -\frac{a}{2} -\rho^\prime\right)^2}{L}\right]\right\}} \; , 
\end{align}
that is, it corresponds to $\psi\left(\rho_1-\frac{a}{2},\rho^\prime\right)$ with an effective refractive index $n_{eff}(\rho^{\prime\prime})=n_1\left(\rho^{\prime\prime}+\frac{a}{2},z^\prime\right)$. Equivalently, we have
\begin{align}
\psi\left(\rho_2,-\rho^\prime+\frac{a}{2}\right) &\propto \int_0^L dz^\prime \int d^2 \rho^{\prime\prime} \frac{n_1(\rho^{\prime\prime},z^\prime)}{z^\prime (L-z^\prime)} \ldots \notag \\
& \hspace{1cm} \times \exp{\left\{\frac{ik}{2}\left[ \frac{(\rho_2-\rho^{\prime\prime})^2}{L-z^\prime} + \frac{\left(\rho^{\prime\prime}+\rho^\prime-\frac{a}{2}\right)^2}{z^\prime} - \frac{\left(\rho_2 +\rho^\prime -\frac{a}{2}\right)^2}{L}\right]\right\}} \notag \\
&=\int_0^L dz^\prime \int d^2 \tau \frac{n_1\left(-\tau+\frac{a}{2},z^\prime\right)}{z^\prime (L-z^\prime)} \ldots \notag \\
& \hspace{1cm} \times \exp{\left\{\frac{ik}{2}\left[ \frac{(\rho_2- \frac{a}{2}+\tau)^2}{L-z^\prime} + \frac{\left(\rho^\prime-\tau\right)^2}{z^\prime} - \frac{\left(\rho_2 -\frac{a}{2} +\rho^\prime\right)^2}{L}\right]\right\}} \; , 
\end{align}
corresponding to $\psi\left(-\rho_2+\frac{a}{2},\rho^\prime\right)$ with an effective refractive index $n_{eff}(\rho^{\prime\prime})=n_1\left(-\rho^{\prime\prime}+\frac{a}{2},z^\prime\right)$.

If we choose detectors $1$ and $2$ to be positioned at $\rho_1=\rho+\frac{a}{2}$ and $\rho_2=-\rho+\frac{a}{2}$, by putting the above equations together we see that both propagators $h$ become identical except for the effective refractive indexes. In this case the propagated two-photon amplitude $E_a\left(\rho+\frac{a}{2}, -\rho+\frac{a}{2}\right)$ turns out to be formally equivalent to a second-order beam with the pump profile and wavenumber (that is, $k_p=2k$) propagating in a medium with refractive index given by
\begin{equation}
n_p(\rho^{\prime\prime},z^\prime)= n_1\left(\rho^{\prime\prime}+\frac{a}{2}, z^\prime\right) + n_1\left(-\rho^{\prime\prime}+\frac{a}{2},z^\prime\right) \; .
\end{equation}

\section{Sampling requirements}
There are three constraints that need to be satisfied to ensure the simulated results accurately represent the corresponding physical scenario. These are:
\begin{subequations}
\begin{align}
\delta_n &\leq \frac{\lambda L - D'_n \delta_1}{D'_1} \; , \label{constraint1} \\
N &\geq \frac{D'_1}{2\delta_1}+\frac{D'_n}{2\delta_n} + \frac{\lambda L}{2\delta_1\delta_n} \; , \label{constraint2} \\
\left( 1+\frac{L}{R}\right) \delta_1 -\frac{\lambda L}{D'_1} & \leq \delta_n \leq  \left( 1+\frac{L}{R}\right) \delta_1 + \frac{\lambda L}{D'_1} \; ,
\end{align}
\label{constraints1to3}
\end{subequations}
where $L$ is the total propagation distance, $R$ is the beam curvature on the source plane, $\lambda$ its wavelength, $N$ is the number os samples taken on each plane in one dimension (that is, grids of size $N\,\times\,N$ are taken), $\delta_1$ and $\delta_n$ are the grid spacing on the source and observation planes and $D'_1$ and $D'_n$ are the linear dimensions of the region of interest on the source and observation planes, respectively, with broadening and blurring caused by atmospheric turbulence taken into account. The turbulence effect is included by adding a term to the actual regions of interest $D_1$ and $D_n$, as follows:
\begin{subequations}
\begin{align}
D_1' &= D_1 +c\frac{\lambda L}{r_{0,rev}} \; , \\
D_n' &= D_n +c\frac{\lambda L}{r_0} \; .
\end{align}
\end{subequations}
The terms $r_0$ and $r_{0,rev}$ are the turbulence coherence diameters as seen from the observation plane and the source plane with counter-propagating light, respectively. These are equal when turbulence is homogeneous, that is, when the structure constant $C_n$ does not depend on the distance $z$ along the propagation axis $Z$. The parameter $c$ is chosen according to what fraction of the emitted light is to impinge in the region of interest. For instance, $c=2$ typically corresponds to about 97$\%$ of captured light while $c=4$ gives $\sim$99$\%$.

Once $N$, $\delta_1$ and $\delta_n$ are chosen, the maximum partial propagation distance should be the following:
\begin{equation}
\Delta z_{max} = \frac{ \left[ \mathrm{min}(\delta_1,\delta_n)\right]^2 N}{\lambda} \; .
\label{constraint4}
\end{equation}
This in turn is associated with a constraint on the minimum number of propagation planes (source and observation planes included):
\begin{equation}
n_{min} = \left\lceil \frac{L}{\Delta z_{max}} \right\rceil + 1 \; .
\end{equation}

We now derive each of the four constraints \eqref{constraints1to3} and \eqref{constraint4}.

\subsection{Constraints 1 and 2}
The constraints described in equations \eqref{constraint1} and \eqref{constraint2} arise from geometric considerations. Given the regions of interest $D_1$ and $D_n$ in, respectively, the source plane $\Pi_1$ and observation plane $\Pi_n$, one has to consider any ray coming from within $D_1$ that might impinge on $D_n$. Among these, the rays propagating from one extreme of $D_1$ to the opposite extreme of $D_n$ are associated with the highest angle $\theta_{max}$ with respect to the propagation axis $Z$ and hence to the largest relevant transverse momentum component $Q_{max}=k\theta_{max}$, where $k=2\pi/\lambda$ is the beam wavenumber.

Now, according to the Fresnel propagation integral, the output field $U_n(\vect{\rho})$ relates to its input $U_1(\vect{\rho}')$ as follows:
\begin{equation}
U_n(\vect{\rho}) = \frac{k e^{ikL}}{2i\pi L} e^{i\frac{k}{2L}\rho^2} \mathcal F \left[\vect{\rho}', \vect{q}=\frac{\rho}{\lambda L} \right] \left\{ U_1(\vect{\rho}') e^{i\frac{k}{2L}\rho'^2} \right\} \; .
\end{equation}
The operator and operand notation $\mathcal F[\vect{\rho},\vect{q}]\{ f(\vect{\rho}) \}$ represents a two-dimensional Fourier transform of the function $f$ from variable $\vect{\rho}$ to $\vect{q}$. Thus, the source of all rays originating from $\Pi_1$ is effectively the field given by $U_1(\vect{\rho}') e^{i\frac{k}{2L}\rho'^2}$, the operand of the Fourier transform. A particular ray can then be represented by a plane wave with transverse momentum $\vect{Q} = \vect{q} + \vect{q}_0(\vect{\rho}')$, where $\vect{q}$ is one component of the input field's angular spectrum and $\vect{q}_0(\vect{\rho}')$ is the local wavevector stemming from $\exp \left[i\frac{k}{2L}\rho'^2\right]$ and given by:
\begin{equation}
\left[\vect{q}_{0}\right]_i = \frac{k}{2L} \frac{\partial \rho'^2}{\partial \rho'_i} = \frac{k\rho'_i}{L} \; .
\end{equation}
In particular, for a ray propagating from edge of $D_1$ to the opposite edge in $D_n$ we have a local transverse wavevector of (in the following analysis we focus on a single transverse component)
\begin{equation}
q_0\left(D_1/2\right) = \frac{kD_1}{2L} \; .
\end{equation}

From figure , one can see that the maximum ray angle is
\begin{equation}
\theta_{max} = \frac{D_1+D_n}{2L} \; ,
\end{equation}
so we have
\begin{align}
q_{max} &= Q_{max} - q_{0\,max} \notag \\
&= k\frac{D_1+D_n}{2L} - k\frac{D_1}{2L} \notag \\
&= \frac{k D_n}{2L} \; .
\end{align}

Because the largest transverse momentum component from the input field $U_1$ affecting the observed field is given by $q_{max}$, the Nyquist criterion dictates the source plane must be sampled at a frequency at least as high as $2\times 2\pi/q_{max}$. That is, the distance $\delta_1$ between sampled points in the source plane must satisfy
\begin{equation}
\delta_1 q_{max} \leq \pi \; ,
\end{equation}
which leads to
\begin{equation}
\delta_1 \leq \frac{\lambda L}{D_n} \; . \label{delta1}
\end{equation}

This criterion prevents what is termed aliasing, an effect by which high spatial frequency components are not properly captured by the sampling procedure. Figure illustrates such an effect: because of a low-sampling frequency, the original signal (black curve) cannot be distinguished from another with smaller spatial frequency (blue curve). The latter is then an alias for the true signal.

Analogously, a properly represented output field will contain sample points spaced closely enough to capture the highest of the relevant spatial frequencies:
\begin{equation}
\delta_n \leq \frac{\lambda L}{D_1} \; \label{deltan} \; .
\end{equation}
Equations \eqref{delta1} and \eqref{deltan} form together the constraint \eqref{constraint1}.

Because the output and input fields are related to each other via a Fourier series (since we need to discretize the Fourier transform), the actual input function becomes a periodic version of the field inside $D_1$. In order to prevent that the repeated versions of the input signal at each of its sides generate a ray that might impinge on $D_n$, the repetition period must be made large enough. 
\end{document}